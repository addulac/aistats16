\section{Homophily: \emph{"Birds of a feather flock together"}}
\label{sec:homophily}
%\vspace{-0.2cm}
%\begin{center} \emph{Birds of a feather flock together} \end{center}
%\vspace{0.1cm}

Homophily refers to the tendency of individuals to connect to similar others: two individuals (and thus their corresponding nodes in a social network) are more likely to be connected if they share common characteristics~\cite{mcpherson2001birds,lazarsfeld1954friendship}. The characteristics often considered are inherent to the individuals: they may represent their social status, their preferences, their interest, ... A related notion is the one of {\it assortativity}, which is slightly more general as it applies to any network, and not just social networks, and refers to the tendency of nodes in networks to be connected to others that are similar in some way.

A definition of homophily has been proposed in~\cite{la2010randomization}. However, this definition, which relies on a single characteristic (as age or gender), does not allow one to assess whether latent models for link prediction capture the homophily effect or not. We thus introduce a new definition of homophily below, which directly aims at this:
%
\begin{definition}[Homophily]
	Let $\mathcal{M}$ be a link prediction model and $s$ a similarity measure between nodes. We say that \emph{$\mathcal{M}$ captures the homophily effect} iff, $\forall (i,j,i',j') \in V^4$:
%
\begin{equation}
\begin{array}{l}
s(i,j) > s(i',j')  \implies \nonumber \\
\pr(y_{ij}=1 \mid \mathcal{M}) > \pr(y_{i'j'}=1  \mid \mathcal{M}) \nonumber
\end{array}
\end{equation}
%
A model which verifies this condition is said to be \emph{homophilic}.
\end{definition}
%
As one can note, this definition directly captures the effect "if two nodes are more similar, then they are more likely to be connected". The similarity function assesses to which extent two nodes share the same (latent) characteristics (or classes). In the case of $\mathcal{M}_e$, these characteristics are captured in the latent features $\mat{F}$ as the model implicitly tries to relate through $\mat{F}$ nodes that are connected in the observation. In the case of $\mathcal{M}_g$ however, there is no mechanism to define latent characteristics as $\mat{F}$ is solely defined from probability distributions that say nothing about possible shared characteristics of nodes. There is thus no sense in this case to talk about homophily. For this reason, we focus on $\mathcal{M}_e$ for homophily. 

The estimated matrix $\mat{\hat{F}}$ captures some latent characteristics of the nodes, whereas the estimated matrix $\mat{\hat{\Phi}}$ captures the correlations between these latent characteristics. One can thus define, on their basis, a "natural" similarity between nodes as follows:
%
\begin{equation}
s(i,j) = \mat{\hat{f}}_{i} \mat{\hat{\Phi}} \mat{\hat{f}}_j^\top \nonumber
%\label{eq:natural-sim}
\end{equation}
%
It is straightforward that $\mathcal{M}_e$ is homophilic wrt to $s$, as $s(i,j)$ corresponds to the probability of generating a link between nodes $i$ and $j$ (Eq.~\ref{eq:link-me}). We thus have the following property:
%
\begin{proposition}[] $\mathcal{M}_e$ is homophilic wrt $s$.
\end{proposition}
%
%\noindent \textbf{Proof} We prove (1), the proof for (1) being similar. We have:
%%
%\begin{equation}
%\begin{array}{l}
%\pr(y_{ij}=1|\mathcal{M}_g) = \int_{\mat{\Phi},\mat{f}_i,\mat{f}_j} \sum_{k,k'} \phi_{k,k'} \pr(z_{i \rightarrow j}=k|\mat{f}_i) \nonumber \\
%\pr(z_{j \leftarrow i}=k'|\mat{f}_j) \pr(\mat{\Phi},\mat{f}_i,\mat{f}_j|\mathcal{M}_g)  d\mat{\Phi}\mat{f}_i\mat{f}_j \nonumber \\
%= \int_{\mat{\Phi},\mat{f}_i,\mat{f}_j} \sum_{k,k'} \phi_{k,k'} f_{ik} f_{jk'} \pr(\mat{\Phi},\mat{f}_i,\mat{f}_j|\mathcal{M}_g)  d\mat{\Phi}\mat{f}_i\mat{f}_j \nonumber \\
%= \int_{\mat{\Phi},\mat{f}_i,\mat{f}_j} \mat{f}_i \mat{\Phi} \mat{f}_j^\top \pr(\mat{\Phi},\mat{f}_i,\mat{f}_j|\mathcal{M}_g)  d\mat{\Phi}\mat{f}_i\mat{f}_j \nonumber \\
%= s_g(i,j) \nonumber
%\end{array}
%\end{equation}
%%
%where $f_{ik}$ denotes the element of $\mat{F}$ at the $i^{th}$ row and $k^{th}$ column and where the second equality stems from the definitions of $z_{i \rightarrow j}$ and $z_{j \leftarrow i}$. This leads to the desired result. \hspace{5.5cm} $\Box$

Hence MMSB in its standard use captures the homophily effect with the "natural" similarity we have considered. We now turn to the preferential attachment effect.