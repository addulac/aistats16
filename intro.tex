\section{Introduction}
\label{sec:introduction}

In recent years, several powerful relational learning models have been proposed to solve the problem commonly referred to as \textit{link prediction} that consists in predicting the likelihood of a future association between two nodes in a network (\cite{Liben-Nowell07, HassanZaki11}). Among such models, the class of probabilistic, generative models has received much attention as such models can be used to both generate artificial networks and infer new links from existing ones. Two main models have been proposed and studied in this class: the latent feature model (\cite{BMF}) and its non-parametric extension (\cite{ILFRM}), and the mixed-membership stochastic block model (\cite{MMSB}) and its non parametric extensions (\cite{iMMSB,diMMSB}). In this paper, we focus on the latter, the mixed-membership stochastic block model (MMSB), and study some of its properties related to link prediction in social networks. 

Indeed, although drawn from a wide range of domains, most real world social networks exhibit common properties, such as the \textit{homophily}, \textit{preferential attachement} and \textit{small world} effects (\cite{Newman2010, Barabasi2003}). 
%
%Homophily is the tendency of people to associate with others having the same characteristics (genre, age \textit{etc.}).  Consequently, when this property is verified in a network,  vertices are more likely to be connected when they are similar than when they are different with regard to their attribute values \cite{mcpherson2001birds}.  Homophily is related to assortativity, that is more general since it refers to the tendency of a node to be linked to others that are similar in some way. On the other hand, preferential attachment, sometimes referred to as the \textit{rich get richer} rule, is a mechanism according to which new nodes prefer to join the more connected nodes existing in the network. Thus, each node is connected to an existing node with a probability that increases with the number of links of the chosen node\footnote{This property is well captured by a power law distribution, hence the claim often made that preferential attachment translates as a power law distribution for the node degrees.}. However, as noted in Leskovec \textit{et al.}, usually in social networks, the entities do not have a global knowledge of the network. The preferential attachment model is thus more likely to be local, and to be specific to communities \cite{LeskovecBKT08}. The small world effect describes the fact that in social networks, one can go from any node in the network to any other node through very few connections.
%
A natural question that arises is thus whether or not generative models as MMSB comply with such properties. More particularly, we address in this study the following two (related) questions:
%
\begin{enumerate}
\item Do mixed-membership stochastic block models comply with the homophily, preferential attachment and small world effects?
\item Do  mixed-membership stochastic block models learned from given observations comply with the homophily, preferential attachment and small world effects?
\end{enumerate}

The remainder of the paper is organized as follows. ???

%In section \label{sec:models}, we present the ILFM model and the IMMSB model in an unified framework which allows to understand their behavior and to compare them. This framework provides notably a better interpretation of the contraints underlying the models regarding the properties of interest. 
 Then, in sections  \ref{sec:homophily} and \ref{sec:burstiness}, we introduce formal definitions of these properties
within the Bayesian framework, and we study how each model
is able to capture this property. We theoretically demonstrate that ILFM is strongly homophilic whereas IMMSB is only weakly homophilic. We show also that ILFM is neutral with respect to global
and local preferential attachment whereas \textbf{IMMSB satisfies the burstiness effect}.  Finally, in section \ref{sec:experiments}, with the purpose to illustrate these theoretical results, we present experiments carried out on real and artificial datasets which confirm the  learning ability of the models on different networks having more or less the properties. Finally, section \ref{sec:concl} concludes this article and gives avenues for future
work. The notations and the mathematical background required in the sequel are briefly recalled in the next section.
 
