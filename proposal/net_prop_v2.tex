\documentclass[a4paper, 12pt]{article}
%\documentclass[a4paper, 12pt, draft]{article} % don't include images, leave a border of same size...great

% Variable de compilation
\newif\ifbeamer
\beamerfalse
\newcommand{\beamer}[2]{\ifbeamer #1 \else #2 \fi}
%%%

%\usepackage[latin1]{inputenc}
\usepackage[utf8]{inputenc} % manage utf8 encodage 
%\usepackage[english]{babel} % for french document ! dirty enumerate style,+ bad change rectangle colors for section linking.
\usepackage{fancyhdr} % for heading
\usepackage{listings}
\usepackage[colorlinks=true, urlcolor=blue]{hyperref} % url, link
\usepackage{graphicx}
\usepackage{geometry}

%\usepackage[cmex10]{amsmath, mathtools}
\usepackage{amsmath,amssymb,amsbsy,amsfonts,amsthm}
\usepackage{multirow}
\usepackage{bm}
\usepackage{enumerate}
\usepackage{url}
\usepackage[ruled,vlined]{algorithm2e}
\usepackage{fancyvrb}
\usepackage{yfonts}

\usepackage{wrapfig}
\usepackage{tikz}
    %\input{../tikz.conf}
    
\usetikzlibrary{bayesnet}
    
%%%%%%%%%%% Box 
\usepackage{calc}%    For the \widthof macro
\usepackage{xparse}%  For \NewDocumentCommand
\newcommand{\tikzmark}[1]{\tikz[overlay,remember picture] \node (#1) {};}

%%%%%%%%%% Math
\renewcommand{\text}{\textnormal}
\newcommand{\pr}{\mathbf{p}}
\newcommand{\E}{\mathbb{E}}
\newcommand{\divkk}{\mathbb{K}}
\newcommand{\entropy}{\mathbb{H}}
\newcommand{\gem}{\mathrm{GEM}}
\newcommand{\Mult}{\mathrm{Mult}}
\newcommand{\DP}{\mathrm{DP}}
\newcommand{\IBP}{\mathrm{IBP}}
\newcommand{\M}{\mathcal{M}}
\newcommand{\V}{\mathcal{V}}
\newcommand{\N}{\mathcal{N}}
    
\makeatletter
\NewDocumentCommand{\DrawBox}{s O{}}{%
    \tikz[overlay,remember picture]{
    	\IfBooleanTF{#1}{%
    		\coordinate (RightPoint) at ($(left |- right)+(\linewidth-\labelsep-\labelwidth,0.0)$);
    	}{%
    	\coordinate (RightPoint) at (right.east);
    }%
    \draw[red,#2]
    ($(left)+(-0.2em,0.9em)$) rectangle
    ($(RightPoint)+(0.2em,-0.3em)$);}
}

\NewDocumentCommand{\DrawBoxWide}{s O{}}{%
	\tikz[overlay,remember picture]{
		\IfBooleanTF{#1}{%
			\coordinate (RightPoint) at ($(left |- right)+(\linewidth-\labelsep-\labelwidth,0.0)$);
		}{%
		\coordinate (RightPoint) at (right.east);
	}%
	\draw[red,#2]
	($(left)+(-\labelwidth,0.9em)$) rectangle
	($(RightPoint)+(0.2em,-0.3em)$);}
}
\makeatother
%%%%% ! Box

\geometry{
      a4paper,
	    body={160mm,260mm},
	    left=25mm,top=20mm,
	    headheight=4mm,headsep=8mm,
        footskip=10mm,
        }
                                              

%%%%%%%%%%%%%%%%%%%%%%%%%%%%%%%%%%%%%%%%%%%%%%%%%%%%%%%%%%%%%%%%%%%%%%%%%%%%%%%%%%%%%%%%%%%%%%%%%%%%%%
%%%%% => Internal
%%%%%%%%%%%%%%%%%%%%%%%%%%%%%%%%%%%%%%%%%%%%%%%%%%%%%%%%%%%%%%%%%%%%%%%%%%%%%%%%%%%%%%%%%%%%%%%%%%%%%%

% itemize item def
%% \begin{itemize}\itemsep2pt % example space betwew item
%\renewcommand{\FrenchLabelItem}{\textbullet}
\renewcommand{\labelitemi}{$\bullet$}
\renewcommand{\labelitemii}{$\cdot$}
\renewcommand{\labelitemiii}{$\diamond$}
\renewcommand{\labelitemiv}{$\ast$}

% equation reference
\renewcommand{\theequation}{\thesection.\arabic{equation}}

%%%%%%%%%%%%%%%%%%%%%%%%%%%%%%%%%%%%%%%%%%%%%%%%%%%%%%%%%%%%%%%%%%%%%%%%%%%%%%%%%%%%%%%%%%%%%%%%%%%%%%
%%%%% => Alias
%%%%%%%%%%%%%%%%%%%%%%%%%%%%%%%%%%%%%%%%%%%%%%%%%%%%%%%%%%%%%%%%%%%%%%%%%%%%%%%%%%%%%%%%%%%%%%%%%%%%%%

% write code
\lstnewenvironment{C}[1]
{\lstset{language=C,
      frame=tBRl,
      basicstyle=\scriptsize,stringstyle=\emph,showstringspaces=false,
      numbers=left,numberstyle=\tiny,
      breaklines=true, columns=flexible, title={#1}}
}{}
      
%%%%%%%%%%%%%%%%%%%%%%%%%%%%%%%%%%%%%%%%%%%%%%%%%%%%%%%%%%%%%%%%%%%%%%%%%%%%%%%%%%%%%%%%%%%%%%%%%%%%%%
%%%%% => Preambles Pages
%%%%%%%%%%%%%%%%%%%%%%%%%%%%%%%%%%%%%%%%%%%%%%%%%%%%%%%%%%%%%%%%%%%%%%%%%%%%%%%%%%%%%%%%%%%%%%%%%%%%%%

\pagestyle{fancy}
\fancyhf{} % remove default headers
\fancyfoot[C]{\thepage}
\renewcommand{\footrulewidth}{0.3pt}
\renewcommand{\headrulewidth}{0.3pt}







\title{
      %\vspace{2cm}
      \Large{Social Networks Properties Study of Latent Class models}\\
            --\\ }

%\date{avril 2015}

\newtheorem{definition}{Definition}[section]
\newtheorem{proposition}{Proposition}[section]
\newtheorem{theorem}{Theorem}[section]

\begin{document}
\maketitle

\section{Introduction}
%Networks properties study of latent class models
%
% # Intro
%
%* generative models encode assumptions about observations data, and
%generative propertie can be stated. Nevertheless, the purpose of such
%latent models is to be infered given the data, which results in a
%modified form of the original canonical expression of the model. This is
%called the posterior distribution.
%
%* We review the Mixed Membership Stochastic Blockmodels Models (MMSB),
%and show that it's design on based conjugate familly gives remarquable
%natural properties that arise in natural networks. Namely the
%preferential attachment effect and the homophily effect, under "non
%convex" combination of models parameters and particularly the mixed
%membership mixing of the models. We study some asymptotic properties of
%a non paramertric version that generalise the models for possibly
%infinite latent classes or blocks and show limitation of the models in
%term of small world effect.
%
%* propose difference between homophily and structural equivalence by
%controling beta -> lead interpretation of class as community....
%
%# related Work / Context
%
%## inference
%Scaling to massive network [], Variational inference and HDP []?. Indian
%buffet process claims better result but harder to be scaled and for same
%reason, hard to assess th properties.
%
%## Blockmodels
%While HDP is used to learn the number of latent classes, few work focus
%on the asymptotic behaviour of such prior and often hyperparameter are
%ignored. We propose here a starting point for such analysis, making
%relatino between the latent class dynamics, with the hyperparameter and
%the observed data.
%
%## preferential attachment and small world. (citation from (Preferential
%Attachment in Graphs with Affnities]
%In order to model the burstiness of network, which is related to
%preferential attachment effect, authors [barabasi, ERGM] has proposed
%several model. But recent works show that model that are infinitely
%exchangeable bayesian graph models (edge or vertex !), are mispecified,
%and "connot model power law" (power is a special case of burstiness...)
%as they lead to dense or empty networks [22]. See Aldous-Hoover theorem,
%see [23]. (similar degeneracie for EGRM models [24, 25].
%
%
%**Where is our work in this scheme**
%
%To overcome this limitation, recent works focus on graphon to models
%some continuity £. [26, 27], but pratical implication are unclear [28].
%


\section{MMSB}
In the following, we will consider graph $G=(V,E)$ of size $|V|=N$. Hence $G$ defines a adjacency binary matrix $Y=(y_{ij})_{i,j \in \mathbb{N}}$ such that $y_{ij}=1$ if there an edge between node $i$ and $j$. Additionally and whitout loss of generality we consider a undirected graph and it imply that $\pr(z_{i\rightarrow j}) = \pr(z_{j\leftarrow i})$ and $\phi_{kk'} = \phi_{k'k}$ \\

...generative models...\\

Basically we have the following independences:
\begin{itemize}
	\item $(y_{ij} \Perp Y^{-ij} \mid \phi_{kk'})$, links interactions are iid given the classes membership and weight interactions.
	\item $(z_{i\rightarrow j} \Perp Y^{-ij} \mid f_i)$,  Each interactions draws it's own membership.
    \item $(\phi_{kk'} \Perp \mat{\Phi}^{-kk'} | \sigma_w)$, class weight interactions are iid (Beta).
    \item $(f_i \Perp F^{-i} | \alpha)$, feature vectors are iid (Dirichlet) over nodes of the network.
\end{itemize}




Notation remark 1: 
we write equivalently  $\pr(z_{i\rightarrow j}=k | f_i)$ as $\pr(k \mid f_i)$ as the probability that node $i$ takes membership in class $k$ for the interactions with $j$. \\

Notation remark 2: $p(d_i^{N, l}) = p((y_{ij})_{j=0}^N )$


\subsection{Likelihoods}
The likelihood that an edge is being generated is given by the observation plate in the graphical model. Marginalizing over the class assignment for both vertices gives us:

\begin{equation} \label{eq:mmsb_like}
p(y_{ij} \mid f_i, f_j, \Phi) = \sum_{k<k'} p(y_{ij} \mid \phi_{kk'})\ p(k\mid f_i)\ p(k'\mid f_j)
\end{equation}

Let's now tackle the probability of the degree $d_i$ of a random vertex $i$, that is equal to $n$. Let's consider the joint probability from the graphical model:
\begin{equation}
p(d_i, F, \Phi \mid \alpha, \lambda) = p(f_i| \alpha)p(\Phi|\lambda) \prod_{j\neq i} p(f_j|\alpha) \sum_{k<k'} p(y_{ij} \mid \phi_{kk'})\ p(k\mid f_i)\ p(k'\mid f_j)
\end{equation}

The degree distribution if then obtained by marginalizing the models parameters:

\begin{equation}
p(d_i \mid \alpha, \lambda) = \int_F \int_\Phi p(f_i| \alpha)p(\Phi|\lambda) \prod_{j\neq i} p(f_j|\alpha) p(y_{ij} \mid f_i, f_j, \Phi) \ dF d\Phi
\end{equation}

As for LDA, there is no closed form expression for this integral due to complex coupling between the weights interactions $\phi_{kk'}$ and the class membership $z_{i\rightarrow .}$. 

\subsection{Predictive Distribution}
Even though the degree distribution has no closed form, the predictive likelihood that a node $i$ generate a \emph{new unobserved edge} $y_{ij}$ given it's degree $d_i$ out of $N_o$ observed edges can be estimated due to the conjugacy of priors within MMSB model. More formally let's first consider that the membership matrix $Z$ associated with the observed degree is known. One can consider the given set $\mathcal{D} = \{d_i, Z\}$. Hence we can rewrite equation \eqref{eq:mmsb_like} as follows:

\begin{align} \label{eq:pred}
p(y_{ij}^{new} \mid \D, \alpha, \lambda) &= \int_{f_i} \int_{f_j} \int_{\Phi} p(f_i|\D, \alpha, ) p(f_j|\D, \alpha)p(\Phi|\D, \lambda, ) \\
&\times \sum_{k<k'} p(y_{ij} \mid \phi_{kk'})\ p(k\mid f_i)p(k'\mid f_j)df_i df_j d\Phi
\end{align}

One can notice the following useful identity here (we omit $\alpha$ here) by noticing that $di \Perp f_i \mid Z$, and that  $f_i | \D$ and  $\Phi | \D$  are respectively  conjugate posteriors in the Dirichlet and beta distributions:
\begin{align}
p(f_i|\D) &= \frac{p(d_i, Z | fi) p(f_i)}{p(z, d_i)} = \frac{p(d_i|Z, f_i) p(Z|f_i) p(f_i)}{ p(d_i|Z)p(Z)} =  \frac{ p(Z|f_i) p(f_i)}{p(Z)} \\
&= Dir(f_i, \mat{\alpha} + \mat{n_i})
\end{align}
Where $\mat{n_i}$ is a k-rows vector representing the count of observed link or non-links such as node $i$ take class $k$, therefore $n_{ik} = \sum_j \bm{1}( z_{i\rightarrow j}=k)$.

Similarly we obtains:
\begin{equation}
p(f_j|\D) = Dir(f_j, \mat{\alpha} + \mat{n_j})
\end{equation}
Hence we have$n_{jk} = \sum_i \bm{1}( z_{i\leftarrow j}=k')$.
Finally for the weight interaction we have that:
\begin{equation}
p(\phi_{kk'}|\D) = B(N_o- d_i^{k,k'} +\lambda_0, d_i^{k,k'} + \lambda_1)
\end{equation}
With $d_i^{kk'} = \sum_j \bm{1}(y_{ij}=1, z_{i\rightarrow j}=k, z_{i\leftarrow j}=k' )$ and the joint distributions of weights interaction is:
\begin{equation}
p(\Phi |\D) = \prod_{k<k'} p(\phi_{kk'}|\D)
\end{equation}

Thus, going back to equation \eqref{eq:pred}, each factorial terms of the posteriors distribution have a term  corresponding to the sum over all $\{k<k'\}_{k, k'=0}^{k,k'=K}$, that can be distributed on. Hence one can recognize that the predictive distribution of generating an edge between $i$ and $j$ is:
\begin{equation}
p(y_{ij}^{new} \mid \D, \alpha, \lambda) = \sum_{k<k'} \E[B((N_o- d_i)^{k,k'} +\lambda_0, d_i^{k,k'} + \lambda_1)]_{kk'} \E[Dir(\mat{\alpha} + \mat{n_i})]_k \E[ Dir(\mat{\alpha} + \mat{n_j})]_{k'}
\end{equation}

Finally we have that:
\begin{equation}
p(y_{ij}^{new}=1 \mid \D, \alpha, \lambda)= \sum_{k<k'} \frac{ d_i^{kk'} + \lambda_1}{N_o^{kk'} + \lambda_{.}} \frac{(n_{ik}+ \alpha)(n_{jk'} + \alpha)}{(N+\alpha_.)^2} =\frac{1}{2} \hat f_i^T \hat \Phi \hat f_j
\end{equation}

We see that the first term account for the node $i$ popularity while the second term account for the similarity between $i$ and $j$, and:
\begin{align}
\hat f_{ik} = \frac{n_{ik}+ \alpha}{N+\alpha_.} \\
\hat f_{jk} = \frac{n_{jk}+ \alpha}{N+\alpha_.} \\
\hat \phi_{kk'} = \frac{ d_i^{kk'} + \lambda_1}{N_o^{kk'} + \lambda_{.}}
\end{align}

One can notice that $d_i^{kk'} = d_i \frac{N_o^{kk',1}}{N_o^{kk'}}=d_i r_{kk'}$. This enable us to extract the degree. Morever let's define the following matrices:

\begin{itemize}
\item $S(i,j) = (\hat f_i \otimes \hat f_j)$ define the node similarity, depending on their class membership and $0<S<1$, \\
\item $C = (\frac{1}{N_o^{kk'} + \lambda_{.}})_{kk'}$, a normalization factor for nodes interactions, \\
\item $R = (r_{kk'})_{kk'}$ express the affinity between the two classes and $0<R<1$.
\end{itemize}

Finally the predictive likelihood can be expressed with those quantities as:

\begin{equation}
p(y_{ij}^{new}=1 \mid \D, \alpha, \lambda)= \frac{1}{2}\ d_i\ sum(R \circ C \circ S(i,j)) + sum(\lambda_1 C\circ S(i,j))
\end{equation}

On the next sections we will study this distribution and analyse their meaning on the structure of generated networks.

%For the general case where $n_j$ is observed greater than 0, we have:
%\begin{equation}
%p(y_{ij}^{new}=1 \mid \D, \alpha, \lambda) = \hat f_i \hat \Phi\hat f_j
%\end{equation}

%$\E_{\Theta|D}[\E_{z_{i\rightarrow j}, z_{i\leftarrow j} | \Theta}[p(y_ij\mid D)]]$



\subsection{Burstiness}


\subsection{Homophily}

\subsection{Small World}

\subsection{Sparsity}

\end{document}
