\section{Preferential attachment: \emph{"The rich get richer"}}
\label{sec:burstiness}

Preferential attachment, sometimes referred to as the \textit{rich get richer} rule, is a mechanism according to which each node is connected to an existing node with a probability that increases with the number of links of the chosen node. In other words, a node is connected to an existing target node with a probability proportional to the number of links of the target node\footnote{This property is well captured by a power law distribution, hence the claim often made that preferential attachment translates as a power law distribution for the node degrees.}. However, as noted in Leskovec \textit{et al.}, usually in social networks, the entities do not have a global knowledge of the network. The preferential attachment model is thus more likely to be local, and to be specific to communities \cite{LeskovecBKT08}.

Preferential attachment relates to a general phenomenon known as \textit{burstiness}\footnote{A.L. Barab\'asi, for example, uses the term \textit{preferential attachement} in \cite{barabasi1999emergence}, and \textit{burstiness} in \cite{barabasi_burst}.} which describes the fact that some events appear in bursts, \textit{i.e.} once they appear, they are more likely to appear again. 
%
%The notion of burstiness is similar to the one of aftereffect of future sampling \cite{feller_68}, which describes the fact that the more we observe an event, the higher the expectation to find new occurrences of this event. In (social) network studies, the burstiness effect is alos referred to as \textit{preferential attachment}: a node with many connections is more likely to have new connections than a node with few connections. To take into account this behavior, in the network generative model  (BA) \cite{albert2002statistical} model, a node is connected to an existing target node with a probability proportional to the number of links of the target node. This leads to scale-free networks that are characterized by a heavy tailed degree distribution, which can be approximated by a power law distribution such that the fraction of nodes $\pr(d)$ having a degree $d$ follows a power law $d^{-\gamma}$, where $\gamma$ typically ranges between 2 and 3~\cite{barabasi1999emergence}. 
%
Burstiness has been studied in different fields, in particular in computational linguistics and information retrieval to characterize word occurrences \cite{church1995poisson}. In these domains, simple definitions of burstiness, that directly capture the fact that a probability distribution is bursty if the probability of generating a new occurrences of an event increases with the number of occurrences of this event, have been proposed\cite{clinchant2008bnb,clinchant2010information}. We rely here on the discrete version of theses definitions, which takes the following form:
%
\begin{definition}[Burstiness]
	A discrete distribution $\pr$ is bursty if and only if, for all integers $(n, n')$ such that $n \geq n'$ :
	\begin{equation}
	\pr(X \geq n+1 \mid X \geq n) > \pr(X \geq n'+1 \mid X \geq n') \nonumber
	\end{equation}
	where $X$ denotes a random variable.
\label{def:burst}
\end{definition}
%
In the context of social networks, the notion of burstiness, or preferential attachement, appears at different levels: (a) a global preferential attachment level that characterizes the degree distribution of nodes in the network, and (b) a local preferential attachment level that characterizes the degree distribution of nodes within classes. The classes we consider here are the latent classes at the basis of the MMSB model.
%, and (c) a class/feature burstiness level that characterizes the distributions of nodes among latent classes/features.
We provide below a formal definition of these elements.
%
\begin{definition}[Preferential attachment in social networks]
Let $i$ be a node in a social network $G=(V,E)$, and let $d_i$ denote its degree. Furthermore, let $\mathcal{M}=\{\mathcal{M}_g, \mathcal{M}_e\}$ be an MMSB model as defined above.
\begin{description}
 \item[(i)] \emph{Global Preferential Attachment}: we say that $\mathcal{M}$ satisfies the global preferential attachement iff, for any node $j \in V$ not connected to $i$, $\pr(y_{ij}=1 \mid d_i \ge n, \mathcal{M})$ increases with $n$.
 \item[(ii)] \emph{Local Preferential Attachment}: we say that $\mathcal{M}$ satisfies the local preferential attachement iff, for any node $j \in V$ not connected to $i$ and belonging to the same latent class $k$ as $i$, $\pr(y_{ij}=1 \mid d_{i,k} \ge n, \mathcal{M})$ increases with $n$; $d_{i,k}$ denotes the degree of node $i$ in the class $k$.
%  \item[(iii)] \emph{Class/Feature Burstiness}: we say that $\mathcal{M}$ satisfies the feature/class burstiness effect, iff, for any node $i$ and latent class/feature $k$, $\pr(f_{ik} > 0 \mid \mat{f}_{\bm{.}k}^{-ik} \ge n, \mat{F}, \mat{\Phi})$ increases with $n$; where $\mat{f}_{\bm{.}k}^{-ik}$ denotes the number of nodes, other than node $i$, assigned to latent class/feature $k$ in the network.
\end{description}
\label{def:burst-soc-net}
\end{definition}
%
The fact that the probability $\pr(y_{ij}=1 \mid d_i \ge n,\mathcal{M})$ increases with $n$ is equivalent to the fact that the probability $\pr(d_{i} \ge n+1 \mid d_i \ge n, \mathcal{M})$ increases with $n$. Indeed:
%
\begin{align}
\pr(d_{i} \ge n+1 \mid d_i \ge n, \mathcal{M}) \nonumber \\
= 1 - \prod_{j \notin \mathcal{V}(i)} P(y_{ij} = 0 \mid d_i \ge n, \mathcal{M}) \nonumber \\
= 1 - \prod_{j \notin \mathcal{V}(i)} (1 - P(y_{ij} = 1 \mid d_i \ge n, \mathcal{M})) \nonumber
\end{align}
%
Hence $\pr(d_{i} \ge n+1 \mid d_i \ge n, \mathcal{M})$ and $P(y_{ij} = 1 \mid d_i \ge n, \mathcal{M})$ vary in the same direction. The same development holds for $\pr(y_{ij}=1 \mid d_{i,k} \ge n, \mathcal{M})$. %The above definitions, characterizing probabilistic link models according to burstiness in social networks, are thus directly related to the general definition of burstiness given in Definition~\ref{def:burst}.

