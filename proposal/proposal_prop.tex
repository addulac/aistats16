\documentclass[a4paper, 12pt]{article}
%\documentclass[a4paper, 12pt, draft]{article} % don't include images, leave a border of same size...great

% Variable de compilation
\newif\ifbeamer
\beamerfalse
\newcommand{\beamer}[2]{\ifbeamer #1 \else #2 \fi}
%%%

%\usepackage[latin1]{inputenc}
\usepackage[utf8]{inputenc} % manage utf8 encodage 
%\usepackage[english]{babel} % for french document ! dirty enumerate style,+ bad change rectangle colors for section linking.
\usepackage{fancyhdr} % for heading
\usepackage{listings}
\usepackage[colorlinks=true, urlcolor=blue]{hyperref} % url, link
\usepackage{graphicx}
\usepackage{geometry}

%\usepackage[cmex10]{amsmath, mathtools}
\usepackage{amsmath,amssymb,amsbsy,amsfonts,amsthm}
\usepackage{multirow}
\usepackage{bm}
\usepackage{enumerate}
\usepackage{url}
\usepackage[ruled,vlined]{algorithm2e}
\usepackage{fancyvrb}
\usepackage{yfonts}

\usepackage{wrapfig}
\usepackage{tikz}
    %\input{../tikz.conf}
    
\usetikzlibrary{bayesnet}
    
%%%%%%%%%%% Box 
\usepackage{calc}%    For the \widthof macro
\usepackage{xparse}%  For \NewDocumentCommand
\newcommand{\tikzmark}[1]{\tikz[overlay,remember picture] \node (#1) {};}

%%%%%%%%%% Math
\renewcommand{\text}{\textnormal}
\newcommand{\pr}{\mathbf{p}}
\newcommand{\E}{\mathbb{E}}
\newcommand{\divkk}{\mathbb{K}}
\newcommand{\entropy}{\mathbb{H}}
\newcommand{\gem}{\mathrm{GEM}}
\newcommand{\Mult}{\mathrm{Mult}}
\newcommand{\DP}{\mathrm{DP}}
\newcommand{\IBP}{\mathrm{IBP}}
\newcommand{\M}{\mathcal{M}}
\newcommand{\V}{\mathcal{V}}
\newcommand{\N}{\mathcal{N}}
    
\makeatletter
\NewDocumentCommand{\DrawBox}{s O{}}{%
    \tikz[overlay,remember picture]{
    	\IfBooleanTF{#1}{%
    		\coordinate (RightPoint) at ($(left |- right)+(\linewidth-\labelsep-\labelwidth,0.0)$);
    	}{%
    	\coordinate (RightPoint) at (right.east);
    }%
    \draw[red,#2]
    ($(left)+(-0.2em,0.9em)$) rectangle
    ($(RightPoint)+(0.2em,-0.3em)$);}
}

\NewDocumentCommand{\DrawBoxWide}{s O{}}{%
	\tikz[overlay,remember picture]{
		\IfBooleanTF{#1}{%
			\coordinate (RightPoint) at ($(left |- right)+(\linewidth-\labelsep-\labelwidth,0.0)$);
		}{%
		\coordinate (RightPoint) at (right.east);
	}%
	\draw[red,#2]
	($(left)+(-\labelwidth,0.9em)$) rectangle
	($(RightPoint)+(0.2em,-0.3em)$);}
}
\makeatother
%%%%% ! Box

\geometry{
      a4paper,
	    body={160mm,260mm},
	    left=25mm,top=20mm,
	    headheight=4mm,headsep=8mm,
        footskip=10mm,
        }
                                              

%%%%%%%%%%%%%%%%%%%%%%%%%%%%%%%%%%%%%%%%%%%%%%%%%%%%%%%%%%%%%%%%%%%%%%%%%%%%%%%%%%%%%%%%%%%%%%%%%%%%%%
%%%%% => Internal
%%%%%%%%%%%%%%%%%%%%%%%%%%%%%%%%%%%%%%%%%%%%%%%%%%%%%%%%%%%%%%%%%%%%%%%%%%%%%%%%%%%%%%%%%%%%%%%%%%%%%%

% itemize item def
%% \begin{itemize}\itemsep2pt % example space betwew item
%\renewcommand{\FrenchLabelItem}{\textbullet}
\renewcommand{\labelitemi}{$\bullet$}
\renewcommand{\labelitemii}{$\cdot$}
\renewcommand{\labelitemiii}{$\diamond$}
\renewcommand{\labelitemiv}{$\ast$}

% equation reference
\renewcommand{\theequation}{\thesection.\arabic{equation}}

%%%%%%%%%%%%%%%%%%%%%%%%%%%%%%%%%%%%%%%%%%%%%%%%%%%%%%%%%%%%%%%%%%%%%%%%%%%%%%%%%%%%%%%%%%%%%%%%%%%%%%
%%%%% => Alias
%%%%%%%%%%%%%%%%%%%%%%%%%%%%%%%%%%%%%%%%%%%%%%%%%%%%%%%%%%%%%%%%%%%%%%%%%%%%%%%%%%%%%%%%%%%%%%%%%%%%%%

% write code
\lstnewenvironment{C}[1]
{\lstset{language=C,
      frame=tBRl,
      basicstyle=\scriptsize,stringstyle=\emph,showstringspaces=false,
      numbers=left,numberstyle=\tiny,
      breaklines=true, columns=flexible, title={#1}}
}{}
      
%%%%%%%%%%%%%%%%%%%%%%%%%%%%%%%%%%%%%%%%%%%%%%%%%%%%%%%%%%%%%%%%%%%%%%%%%%%%%%%%%%%%%%%%%%%%%%%%%%%%%%
%%%%% => Preambles Pages
%%%%%%%%%%%%%%%%%%%%%%%%%%%%%%%%%%%%%%%%%%%%%%%%%%%%%%%%%%%%%%%%%%%%%%%%%%%%%%%%%%%%%%%%%%%%%%%%%%%%%%

\pagestyle{fancy}
\fancyhf{} % remove default headers
\fancyfoot[C]{\thepage}
\renewcommand{\footrulewidth}{0.3pt}
\renewcommand{\headrulewidth}{0.3pt}







\title{
	%\vspace{2cm}
	\Large{Social Networks Properties Study of Latent Class models}\\
	--\\ }

%\date{avril 2015}

\newtheorem{definition}{Definition}[section]
\newtheorem{proposition}{Proposition}[section]
\newtheorem{theorem}{Theorem}[section]

\begin{document}

\section{Models and Properties}
Nous définissons un modèle génératif selon 3 objets:
\begin{itemize}
\item Les hyperparamètres d'un modèle $\mathcal{M}$
\item les paramètres aléatoires du modèle $\Theta$. Dans les latent class models et particulièrement MMSB, on a $\Theta = \{F, \Phi, Z\}$
\item Les observations Y qui sont les objets générés par le modèle.
\end{itemize}

Notre but est d'étudier le comportement de modèle génératif de réseaux. Pour cela on définit des propriétés adaptés à un cadre probabiliste. Nous affirmons qu'une propriété peut être considéré selon deux hypothèse distincte:
\begin{itemize}
\item propriété strong: propriété vrai et inhérente au modèle, on regarde ce que génère un réseaux étant donné ses paramètre aléatoires (intérêt pour l'ingénierie des paramètres...). $\Theta \longrightarrow Y$
\item propriété weak: on ne connaît pas les paramètre aléatoires, seulement, les hyperparamètres. On raisonne selon ce schéma $\mathcal{M} \longrightarrow Y$
\end{itemize}

\paragraph{prosition1:}
Si une propriété est strong, alors elle est aussi "weak" vrai, car si elle est vrai pour un choix arbitraire de paramètres, ce choix tombe dans le tirage possible du modèles $\mathcal{M}$.\\

Dans la suite de cet écrit, nous proposons/définissons et vérifions le respect de propriétés fondamental des réseaux(preferantial attach, homophily, small world, sparsity).

\section{MMSB}

...

\section{Burstiness $\Delta_n p(y_ij=1|d_i>n) > 0 $}

i) Le réseau est trivialement **non strongly bursty**\\
ii) weakly ? 

Mon raisonnement est le suivant:
\begin{enumerate}

\item on constate que la distribution du degrée n'est pas exploitable et n'admet pas de forme close connue (see Neal)
\item on montre qu'en générant $Z$  (ie $Z$ est généré avec $d_i$, par le modèle), on a une expression [1].
\item on montre que cette expression permet de réécrire le burstiness
\item on montre que le burstiness est croissant si [1] est croissant
\item on montre que [1] est croissant avec n quelque soit les concentrations données par Z...
\end{enumerate}

De la définition du burstiness on obtient:
\begin{equation} 
p(y_{ij} = 1 | d_i>n, \mathcal{M}) = \int_{\Theta} p(y_{ij}=1|\Theta) \frac{p(d_i>n | \Theta)}{p(d_i>n)} p(\Theta) d\Theta
\end{equation}

Dans [*] on montre que distribution sur le degré n'a pas de forme connue (fait bien connue pour le p(w) de LDA).
Par contre, nous pouvons faire l'hypothèse que pour chaque valeur possible $l$  du degré généré, soit accompagné de la matrice Z contenant les valeurs des classe d'attributions, pour chaque lien généré. Notons que nous ne faisons aucune hypothèse sur la distribution des classes d'attributions (c'est à dire de leur concentration, on dit simplement que chaque liens (ou non lien) ets associé à un couple kk'). la distribution est désormais implicitement conditionné par Z. Hors, étant donnée que le modèle est jointement échangeable [aldous, Hoff], on obtient:

\begin{align} \label{eq:sum}
p(y_{ij} = 1 | d_i>n, \mathcal{M}) &= \int_{\Theta} p(y_{ij}=1|\Theta) \frac{\sum_{l=n}^N\dbinom{N}{l} p(d_i^N=l | \Theta)}{\int_{\Theta}\sum_{l=n}^N \dbinom{N}{l}p(d_i^N=l | \Theta)} p(\Theta) d\Theta \\
&= \frac{\sum_l^N\dbinom{N}{l} \int_{\Theta} p(y_{ij}=1|\Theta) p(d_i^N=l | \Theta) p(\Theta) d\Theta }{\sum_l^N \dbinom{N}{l}\int_{\Theta} p(d_i^N=l | \Theta) p(\Theta) d\Theta } \\
&=  \frac{\sum_l^N a_l}{\sum_l^N b_l} 
\end{align}

Interestingly, one can recognize that:
\begin{equation}
\frac{a_l}{b_l} = p(y_{ij}=1 | d_i^N=l, Z) 
\end{equation}

We show in [*] that for an undirected network, the following holds:

\begin{align}\label{eq:fu}
pp(y_{ij}=1 | d_i^N=l, Z) &=\sum_{k<k'} \frac{ d_i^{kk'} + \lambda_1}{N_o^{kk'} + \lambda_{.}} \frac{(n_{ik}+ \alpha)(n_{jk'} + \alpha)}{(N+\alpha_.)^2} \\
&=\frac{1}{2} \hat f_i^T \hat \Phi \hat f_j
\end{align}

We show that an other interesting parametrization, higlithing the degree dependance is:


\begin{equation} \label{eq:fu}
p(y_{ij}=1 | d_i^N=l, Z) = \frac{1}{2}\ l\ sum(R \circ C \circ S(i,j)) + sum(\lambda_1 C\circ S(i,j))
\end{equation}

Where $R$, $C$ and $S$ are strictly positive matrix and their $sum(.)<1$. Moreover those matrix only depend on the class concentration whitin each node and more precisely see [*]:
\begin{itemize}
\item $S(i,j) = (\hat f_i \otimes \hat f_j)$ define the node similarity, depending on their class membership and $0<S<1$, \\
\item $C = (\frac{1}{N_o^{kk'} + \lambda_{.}})_{kk'}$, a normalization factor for nodes interactions, \\
\item $R = (r_{kk'})_{kk'}$ express the affinity between the two classes and $0<R<1$.
\end{itemize}


Thus one can write that $a_l = b_l g_l$, with $g_l$ defined by \eqref{eq:fu}.

Now let's define the following series:
\begin{equation}
an = \frac{\sum_{l=n}^N b_l g_l}{\sum_{l=n}^N b_l}
\end{equation}

Given that $b_l$ and $g_l$ are positive, we show that if $\Delta_n g_n$ is increasing with $n$, then it implies that $a_n$ is increasing with $n$:

proof:
\begin{align}
\Delta_n a_n &= \frac{\sum_{l=n+1}^N g_l b(l)}{\sum_{l=n+1}^N b(l)} - \frac{g_n b(n) + \sum_{l=n+1}^N g_l b(l)}{b(n) + \sum_{l=n+1}^N b(l)} \\
&= \frac{b(n)}{\left(b(n) + \sum_{l=n+1}^N b(l)\right) \left(\sum_{l=n+1}^N b(l)\right)} \sum_{l=n+1}^N (g_l-g_n)b(l)
\end{align}
As we have $l \geq n$, it proves the hold.\\

Because $g_l$ is linear\footnote{Note that this linearity is broken if the generation of new links would modify the concentration of links which is not the case in the assumptions of MMSB.} at a fixed class contration state (corresponding to matrix Z) with the degree, it is straightforward  that the networks satisfies the global preferential attachment.

Similarly, for the local preferential attachment we can assess that from the definition, when (i) and (j) are in class c={k,k'}, we have simply that:
\begin{equation}
p(y_{ij} | d_{ic}^N=l, Z) = \frac{1}{2}\ l\ R_c  C_c S(i,j)_c + \lambda_1 C_c S(i,j)_c
\end{equation}

Again, we see that local preferential burstiness is true, furthermore as $ R_c  C_c  S(i,j)_c) + \lambda_1 C_c S(i,j)_c$ is a strictly increasing function of concentration of class assignment in $c$.

\subsection{Discussion}


* local preferential attachment inter-commmunity!! (homophily vs stochastic equivalence). Pure power law function not possible (see expe)\\
* impact of global density\\
* Haldous hoover theorem and temporal networks sparsity and limit.
* Relation with diaconis-Ylvisaker theorem, and how we generalize it, for mixed membershi model in exponential family. (probabilistic\_burstiness)

\section{Smal World}\label{smal-world}

The small world effect is the property that exhibits many real networks
such as social networks and the Web. This result by making any nodes of
the network connected to any others by a path of small length. One
possible definition is as follows; given  two randomly chosen nodes \((ij)\), the
expected shorted distance \(L\) between them grows proportionally with the logarithm
of the number of nodes \(N\) in the network, that is:

\begin{equation}
L \propto \log(N)
\end{equation}

In this formalism scale-free networks are know to be ultra-small
networks, with the shortest distance scale as \cite{cohen2003scale}:

\begin{equation}
L \propto \log\log(N)
\end{equation}

However, The notion of shortest distance is not an easy quantity to define in a
random graph. However we defined the notion of latent class previously,
where class define as a variable where links can occur inside (or
outside). Hence one can have the
intuition that if links are mostly created in a set of class (inner or
outer), the shortest path between two random node is
``proportional'' to the number of step needed  to go through all classes.

In a case of iMMSB, the number of class is unknown and evolves with the data
through an HDP prior. The question that we ask here, is how this number
evolve with the size of the networks ie its number of vertices.

In the HDP {[}WTEH{]}, we have two step DP that form a so called Chinese
Restaurant Franchise (CRF) with infinite random component being:
\begin{itemize}
\item the number of
table in each restaurant, where restaurant symbolically correspond to
nodes and tables are each links (and non links) of this node. 
\item the
number of class in the franchise, where each table is associate to
couple of classes.
\end{itemize}

By definition of a \(DP(\alpha)\), we know that for each new draw given
\(i\) previous draws, the probability to generate a new group is equal
to \(\frac{\alpha}{\alpha +i-1}\) indepently of the number of previous
cluster. Moreover in the CRF, each table is draw from a DP while the
classes for each tables are drawn fom one another DP (shared in all the
franchise) . Thus we can express the expectation of the number of tables
($\E[T|N, \alpha_0]$) and class ($\E[K|T, \gamma]$). We start by the first quantity, with a graph of size $N >> 0$:

\begin{align}
\E[T|N, \alpha_0] = \sum_{j=1}^N\sum_{i=1}^N \frac{\alpha_0}{\alpha_0 +i-1} = \alpha_0 N (\Psi(\alpha_0+N) - \Psi(\alpha_0))
\end{align}
Given the properties of the $\Psi$ function, we have three different limit for $E[T|N]$:
\begin{itemize}
	\item $\alpha_0 N log(1+\frac{N}{\alpha_0})$ if  $\alpha_0 \sim 0$
	\item $\alpha_0 N log(N)$ if  $\alpha_0 \sim 1$
\end{itemize}
Here, we omit the case of $\alpha_0 >> 0$ as non usefull in real case because small value is preferred since it offers a better sparsity about the proportion of class that node belong to.
Similarly, for the expectation of the number of class, we have:

\begin{equation}
\E[K|T, \gamma] = \sum_{i=1}^N\sum_{i=1}^{2T} \frac{\gamma}{\gamma +i-1} = \gamma (\Psi(\gamma+2T) - \Psi(\gamma))
\end{equation}

Finally one can note that $N >> 0$ implies that $T>>0$, hence we gives here a table of asymptotic value for $E_{T|\alpha_0}[E[K|T, \gamma]$:
\begin{table}[h]
	\begin{tabular}{c|cc}
		 & $\alpha_O \sim 0$&$\alpha_O \sim 1$ \\
		\hline
		$\gamma \sim 0 (or \infty)$ & $2\gamma log(\frac{\alpha_0Nlog(N/\alpha_0)}{\gamma})$ & $2\gamma log(\frac{\alpha_0Nlog(N)}{\gamma})$ \\
		$\gamma \sim 1$ & $2\gamma log(\alpha_0Nlog(N/\alpha_0))$ &  $2\gamma log(\alpha_0Nlog(N))$
	\end{tabular}
\end{table}

We conclude that the model has some good affinities with the small world effect has the number of class grow in $log(log(N))$ if the paramater $\gamma$ has decay lesser than $exp(-N)$.


\section{Sparsity}
(defnition of dense and sparse netwroks are given in [Haldous])

In the model the density for undirected networks of size $|V|=N$ is defined by the expectation of generating a links:
\begin{equation}
p(y_{ij}=1| |V|=N) =  \int_{\Theta} p(y_{ij}=1|\Theta) \frac{\sum_{n=0}^{\dbinom{N}{2}} p(|V|=n | \Theta)}{\int_{\Theta}\sum_{n=0}^{\dbinom{N}{2}}p(|V|=n | \Theta)} p(\Theta) d\Theta
\end{equation}

again the joint excheangibility of the models gives us a similar result than equation \eqref{eq:sum}.

Thus it prooves that the density is increasing with n, and thus that the number of edge scale at least in $N^2$. 

It shows that the model generate dense networks because the model is mispecified. This result in a direct consequence of the theorem of Haldous for exchangeable graph.


\end{document}
